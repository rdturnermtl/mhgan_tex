\begin{figure}[htbp]
    \centering
    \begin{subfigure}[b]{0.32\textwidth}
       \centering
       \includegraphics[width=1.8in]{figures/per_epoch.pdf}
       \caption{performance by epoch}
       \label{fig:incep_by_epoch}
    \end{subfigure}
    \begin{subfigure}[b]{0.32\textwidth}
       \centering
       \includegraphics[width=1.8in]{figures/plot_per_mh.pdf}
       \caption{performance by MCMC iteration}
       \label{fig:incep_by_iter}
    \end{subfigure}
    \begin{subfigure}[b]{0.32\textwidth}
       \centering
       \includegraphics[width=1.8in]{figures/score_dist_bta.pdf}
       \caption{epoch 13 scores}
       \label{fig:score_dist_overlap}
    \end{subfigure}
    \caption{{\small
    Results of the MH-GAN experiments on CIFAR-10 using the DCGAN\@.
    On the left, we show the inception score by training epoch of the DCGAN at $K=640$.
    MH-GAN denotes using the raw discriminator scores and ``MH-GAN (cal)'' for the calibrated scores.
    The error bars on MH-GAN performance (in gray) are computed using a t-test on the variation per batch across 80 splits of the inception score.
    In the center we show the inception score vs.~MCMC iteration $k$ for the GAN at epoch 15.
    On the right, we show the scores at epoch 13 where there is some overlap between the scores of fake and real images.
    When there is overlap, the MH-GAN corrects the $\PG$ distribution to have scores looking similar to the real data.
    DRS fails to fully shift the distribution because 1)~it does not use calibration and 2)~its ``$\gamma$ shift'' setup violates the validity of rejection sampling.
    }}
    \label{fig:inception}
\end{figure}

